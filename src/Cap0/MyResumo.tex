%-----------------------------------------------------------------------------------------------------------------
% Resumo em Português
%-----------------------------------------------------------------------------------------------------------------



\begin{center}
\large{Pipeline de testes automatizados para integração e entrega contínua de software B2B em desenvolvimento Agile}


\vskip5mm 
 
\end{center}

\textbf{Resumo ---} 
No desenvolvimento de software de acordo com a metodologia \textit{Agile}, a satisfação do cliente é a principal prioridade. Assim sendo, a entrega contínua de software funcional, compatível com os requisitos e em formatos confiáveis, é o que caracteriza esta técnica de gestão de projetos.\\ 

\hspace{1cm}Uma vez que os objetivos da garantia de qualidade vão de encontro aos pressupostos dos princípios do manifesto \textit{Agile}, a implementação de uma \textit{pipeline} de integração e entrega contínua pode ser uma das soluções para dar resposta às necessidade de adaptação a constantes mudanças sentidas pelas empresas. Nos casos em que este tipo de práticas são comuns verifica-se efetivamente um aumento da resiliência.\\ 

\hspace{1cm}A existência de uma \textit{pipeline} de entrega contínua é bastante benéfica na medida em que permite possuir um ecossistema de apoio ao desenvolvimento, com um sistema de controlo de versões, um orquestrador de processos que, por sua vez, vai coordenar uma ferramenta de integração contínua que realiza análise estática, testes unitários e testes de integração. Posteriormente, um sistema de transição de estados, atualiza as fases dos projetos de \textit{development} para \textit{staging}, \textit{pre-live} e \textit{live}. Todas estas ferramentas e automatizações têm como principais objetivos o suporte no desenvolvimento de software com qualidade, a otimização do tempo de desenvolvimento e, sempre que acontecem problemas, o apoio na tomada de decisão para aumentar a celeridade da resposta.\\

\textbf{Keywords:} Quality assurance, Integração contínua, Entrega contínua, DevOps, Gestão de projetos, Agile.

%-----------------------------------------------------------------------------------------------------------------
